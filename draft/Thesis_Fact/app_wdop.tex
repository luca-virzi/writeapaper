\chapter{$\mathcal{O}(a)$-improved Wilson-Dirac operator}
\label{app: Wilson improved op}

The massive $\mathcal{O}(a)$-improved Wilson-Dirac operator is defined as \cite{L_scher_2013, L_scher_1996, SHEIKHOLESLAMI1985572}:
\begin{equation}\label{dirac improved}
    D = D_{\textrm{w}} + \delta D_{\textrm{v}} + \delta D_{\textrm{b}} + m_0
\end{equation}
where $m_0$ is the bare quark mass, and $D_{\textrm{w}}$ is the massless Wilson-Dirac operator:
\begin{equation}
    D_{\textrm{w}} = \frac{1}{2}\{\gamma_\mu \left(\nabla^*_\mu + \nabla_\mu \right) - \nabla^*_\mu \nabla_\mu\}
\end{equation}
with $\gamma_\mu$ being the Dirac matrices, and the summation over repeated indices is understood.
\\ We define the covariant forward and backward derivatives $\nabla_\mu$ and $\nabla^*_\mu$ as:
\begin{equation}
    \nabla_\mu \psi(x) = U_\mu(x) \psi(x + \hat{\mu}) - \psi(x) \hspace{8mm} \nabla^*_\mu \psi(x) = \psi(x) -  U^\dagger_\mu(x - \hat{\mu}) \psi(x - \hat{\mu}) - \psi(x)
\end{equation}
where $\psi(x)$ is a generic fermion field, $U_\mu(x)$ are the link fields and $\hat{\mu}$ is the unit vector defined along the direction $\mu$.
The other terms that appear in \eqref{dirac improved} are called boundary correction terms, and they are defined as:
\begin{equation}
    \begin{split}
       & \delta D_{\textrm{v}} \psi(x) = c_{\textrm{SW}} \frac{i}{4} \sigma_{\mu\nu} \hat{F}_{\mu\nu}(x) \psi(x) \\
        & \delta D_{\textrm{b}} \psi(x) = \{ (c_{\textrm{F}} - 1)\delta_{x_0, 1} + c'_{\textrm{F}} - 1)\delta_{x_0, T-1} \} \psi(x)
    \end{split}
\end{equation}
and with open boundary conditions in the time direction, $c_{\textrm{F}} = c'_{\textrm{F}}$. In the previous expression we defined the tensorial structure $\sigma_{\mu\nu} = \frac{i}{2}\left[\gamma_\mu, \gamma_\nu \right]$, and $\hat{F}_{\mu\nu}$, which is a discretization of the field strength tensor given by:
\begin{equation}
    \hat{F}_{\mu\nu} = \frac{1}{8} \{ Q_{\mu\nu}(x) - Q_{\nu\mu}(x)\}
\end{equation}
with:
\begin{equation}
    \begin{split}
        Q_{\mu\nu}(x) & = U_\mu(x) U_\nu(x + \hat{\mu}) U_\mu^\dagger(x + \hat{\nu}) U_\nu^\dagger(x) + \\
        & + U_\nu(x) U^\dagger_\mu(x - \hat{\mu} + \hat{\nu}) U_\nu^\dagger(x - \hat{\mu}) U_\mu(x - \hat{\mu}) + \\
        & +  U^\dagger_\mu(x - \hat{\mu}) U^\dagger_\nu(x - \hat{\mu} - \hat{\nu}) U_\mu(x - \hat{\mu} - \hat{\nu}) U_\nu(x - \hat{\nu}) + \\
        & +  U^\dagger_\nu(x - \hat{\nu}) U_\mu(x - \hat{\nu}) U_\nu(x + \hat{\mu} - \hat{\nu}) U^\dagger_\mu(x)
    \end{split}
\end{equation}
In the block decomposition of the Dirac operator, we will refer to $Q = \gamma_5 D$, which is Hermitian since $D$ satisfies the $\gamma_5$-hermiticity relation $D = \gamma_5 D^\dagger \gamma_5$.