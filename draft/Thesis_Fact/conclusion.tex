\chapter*{Conclusions and outlook}

The Schwinger Model revealed itself as an instructive framework for several reasons. The design of an Hybrid Monte Carlo algorithm was a good exercise to get acquainted with the numerical challenges one may encounter in Lattice Gauge Theories, and since the algorithm was constructed from scratches, both theoretical and computational aspects have been taken into account. Hadron spectroscopy in QED$_2$ shares several features with QCD$_4$, therefore it was interesting to study the low-lying spectrum of this theory and compare the results with the theoretical predictions. For what concerns the iso-triplet states of the spectrum, the correct approach to chiral limit was successfully retrieved, as the pseudoscalar mass vanished when we drove the fermion mass to zero.
The study of the iso-singlet state, on the other hand, was definitely more challenging, due to fact that its two-point function takes contribution from disconnected Wick contractions, which require way higher computational resources in order to produce accurate data. Nevertheless, we produced valuable results for the correlation function, and also the $\eta$-meson behaviour towards the chiral limit has been roughly confirmed. 
On a broader perspective with respect to this work, deeper analysis may be conducted on this model, as its computational costs are sufficiently low with respect to Lattice QCD.
\\ Despite being somewhat interesting and stimulating by itself, we turned our attention to the Standard Schwinger Model in order to collect some results which could be a useful benchmark for a future algorithm where the new factorization of the fermionic determinant will be employed. The deeper goal of this work is indeed inspired by the perspective of finding new computational methods for Lattice Gauge Theories.
\\ At the state of the art, the current factorization procedure is one of soundest ways to tackle down the exponential problem which affects most computations in Lattice Field Theories, as it paved the way for the implementation of multilevel algorithms for systems with dynamical fermions (there comes the decision to briefly present it here).
\\ Nevertheless, the aim of this work consists in the proposal of a new technique for the factorization of the fermionic determinant, which from our point of view represents something unique in literature, as it is the first procedure where a complete factorization is realized. The decomposition of the lattice in overlapping domains was abandoned in favor of a full block-separation of the system dynamics. The theoretical realization of this procedure is inspired by the fact that the Dirac-Wilson operator can be written as a sparse block matrix, which in turn implies the possibility to exploit algebraic resolution methods to address our problem. Whether in both factorization procedures it is somewhat reasonable to separate most of the contributions to the fermionic determinant into terms which are local in the block fields, only in our proposal the factorization is complete, while the current technique relies on the introduction of multiboson auxiliary fields in order to rewrite the non-local remainder of the quark determinant as a bosonic effective action.
\\ Although one may have preferred to test immediately the effectiveness of this new technique, before we can actually implement it numerically we first need to formalize the factorization of fermionic observables, which nevertheless seems to be rather natural with this prescription and will be the next step in this research line. Consequently, we will have all the necessary ingredients to design multi-level algorithms for systems with dynamical fermions where our new proposal gets implemented.
\\ In order to offer a proper support to the validity of our findings, we tested our factorization prescription numerically, and the positive outcome of these investigations boosts our confidence in the potentiality of this new procedure, even though only some proper computational tests will tell us whether its efficiency is comparable with the current methods in literature.